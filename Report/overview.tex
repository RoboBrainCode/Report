% !TEX root = main.tex
\section{Problem Overview}

An end-to-end robotic system consist of cascade of modules which are designed to enable specific robotic abilities such as grasping, path planning, language understanding etc. In this
work we address the problem of learning such end-to-end systems in an interactive manner with a human-in-the-loop. We want the interactive learning procedure to be easy for any non-expert user to use. Such non-expert users are typically agnostic to the individual modules involved in the overall robotic system and therefore cannot manually tune each module. In fact they can at best judge the final behaviour produced by the robot. We propose an interactive procedure, where by simply observing user preferences on the final output produced by the robot an end-to-end robotic system is learned.

In our formulation we consider cascade of modules that accomplish a robotic task, where the output of a module forms the input to the module following it. The input into the cascade is a robotic task which is then forward propagated through multiple modules. The last module in the cascade generates the final robot behaviour which is visible to the user e.g. a trajectory accomplishing the task. In our learning setting, we elicit a \textit{coactive feedback} from the user on the final robot behaviour. Through this feedback the user slightly improves upon the output produced by the robot, but never reveals the optimal behaviour. Our feedback mechanism has certain desirable characteristics: (i) it is \textit{weak}, in the sense the non-expert user never reveals the optimal behaviour to the robot; and (ii) it is \textit{global}, in the sense the user does not provide feedback at the module level and only improves upon the final robot behaviour. The \textit{global} nature of feedback makes learning challenging since the user feedback needs to be back-
propagated to each module. We now formally define the learning setting.

\section{System Overview}
\todo{Dip: I vote for this section to come later. Model.tex is the main meat and we should try to
bring it up as soon as possible. This section is mere details}
System consists of web based interface where user can see the robotic tasks
executed for a particular NLP instruction. User can give their like and dislike
over this output. User can provide more detailed expert feedback upon it and it
can be used to evaluate our system.  \todo{Dip: This is not strictly correct. Our feedback
requires $\bar{y}$ to be of same type as $y$. Our problem formulation does not work with
like-dislike}


